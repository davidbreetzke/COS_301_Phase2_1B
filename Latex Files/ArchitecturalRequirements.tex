\documentclass[11pt]{article}
\usepackage{graphicx} %Required for diagrams
\usepackage[bookmarks=true]{hyperref}
\usepackage{bookmark}%Required to do pdf bookmarking
\usepackage{hyperref}%Required for referencing website pages

\begin{document}
\begin{titlepage}
\begin{center}
\includegraphics[width=350px]{University_of_Pretoria_Logo.png}
\newline
\newline


\begin{flushright} \large
\textbf {\textsc{\LARGE COS301}}\newline
\textbf {\textsc{\LARGE Mini Project}}\newline
\textbf {\textsc{\LARGE Architectural Requirements Specification}}\newline
\end{flushright}



%\begin{minipage}{0.4\textwidth}
\textbf{Group 1B} \\
\begin{flushright} \large
David Breetzke \emph{u12056503} \newline
Tienie Pritchard \emph{u12056741} \newline
Ephiphania Munava \emph{u10624610} \newline
Paul Engelke \emph{u13093500} \newline
New Member \emph{uxxxxxxxx} \newline
New Member \emph{uxxxxxxxx} \newline
New Member \emph{uxxxxxxxx} \newline
New Member \emph{uxxxxxxxx} \newline 
\newline 
\newline
\end{flushright}
%\end{minipage}
GitHub: \href{https://github.com/davidbreetzke/COS_301_Phase2_1B}{Visit}
\vfill
{\large Version 1}
\\
{\large \today}

\end{center}
\end{titlepage}


\tableofcontents	%Creates Table of contents from sections and subsections, etc...
\newpage

%Access and Integration Requirements
	\newpage
	\begin{center}
	\section{\textbf{\huge{Access and Integration Requirements}}}
	\end{center}
	
	[insert text]
%---------------------------------------------------------

%Access Channels
	\newpage
	\begin{center}
	\section{\textbf{\huge{Access Channels}}}
	\end{center}
	
	[insert text]
%---------------------------------------------------------
		
%Integration Channels
	\newpage
	\begin{center}
	\section{\textbf{\huge{Integration Channels}}}
	\end{center}
	
	[insert text]
%---------------------------------------------------------

%Architectural Responsibilities
	\newpage
	\begin{center}
	\section{\textbf{\huge{Architectural Responsibilities}}}
	\end{center}
	
	[insert text]
%---------------------------------------------------------

%Quality Requirements
	\newpage
	\begin{center}
	\section{\textbf{\huge{Quality Requirements}}}
	\end{center}
%---------------------------------------------------------
	\subsection{Scalability}
		[insert text]
	\subsection{Performance Requirements}
		[insert text]
	\subsection{Maintainability}
		[insert text]
	\subsection{Reliability and Availability}
		[insert text]
	\subsection{Security}
		[insert text]
	\subsection{Monitorability and Auditability}
		[insert text]
	
	\subsection{Scalability}	
		[insert text]
	
	\subsection{Testability}
	
		\subsubsection{Type of Quality:}
			\textbf{} System Quality
		
		\subsubsection{Priority:}
		\textbf{} Critical
		
		\subsubsection{Description:}
		\textbf{}Testability measures how easy it is to create testing standards for a system and its individual components, theses standards are tested to evaluate if a criteria has been met. Thus software testability is the point to which the software system supports testing in some context. Hence if the software testability is high finding faults in the system is easier.
		
		\subsubsection{Stake Holder:}
		\begin{itemize}
			\item Persons who operates the system : Administrator, Maintenance Operator and Tech-team.
			\item Persons who benefits from the system : Lectures, Teaching Assistance, Tutors, Students and Guest.
			\end{itemize}
		\subsubsection{Context:}
		\begin{itemize}
			\item Stimulus : The testing is performed by tester (these might be system testers, integration testers and even the end user).
			\item Artifact : The target of the attack can be the system or the data in the system.
			\item Environment – This attack can come from the user of the system or an outsider like a hacker. 
			\item Response : The system has to authorize certain actions and responses for each of the given tasks.
			\item Response Measure : The measure of the system and it functionality before, during and after the attack.
		\end{itemize}
		
		
		\subsubsection{Measurable Specification:}
		\begin{itemize}
			\item[•] Understand-ability : The point at which the component of the system that being tested is self-explanatory.
			\item Separation of concerns : The point, at which the component of the system that's being tested has a well-defined responsibility.
			\item Observe-ability : The point, at which the component of the system that’s being tested become possible to discern the test results. \\
						
			\textbf{Component Under Test}\\
			Is a test that restrictions the scope of the used software to a ration of the system that is being tested.
			
			\item Controllability : The point, at which the system that’s being tested becomes possible to control the state of the component under test as required.
			\item Isolate-ability : The point, at which the system that’s being tested becomes possible for the component under test to be tested in isolation.
			
		\end{itemize}
		
		
	\subsection{Usability }
		[insert text]
	\subsection{Integrability}
		[insert text]

%Architecture Constraints
	\newpage
	\begin{center}
	\section{\textbf{\huge{Architecture Constraints}}}
	\end{center}
	
	[insert text]
%---------------------------------------------------------
\end{document}