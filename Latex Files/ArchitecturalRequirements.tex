\documentclass[11pt]{article}
\usepackage{graphicx} %Required for diagrams
\usepackage[bookmarks=true]{hyperref}
\usepackage{bookmark}%Required to do pdf bookmarking
\usepackage{hyperref}%Required for referencing website pages

\begin{document}
\begin{titlepage}
\begin{center}
\includegraphics[width=350px]{University_of_Pretoria_Logo.png}
\newline
\newline


\begin{flushright} \large
\textbf {\textsc{\LARGE COS301}}\newline
\textbf {\textsc{\LARGE Mini Project}}\newline
\textbf {\textsc{\LARGE Architectural Requirements Specification}}\newline
\end{flushright}



%\begin{minipage}{0.4\textwidth}
\textbf{Group 1B} \\
\begin{flushright} \large
David Breetzke \emph{u12056503} \newline
Tienie Pritchard \emph{u12056741} \newline
Ephiphania Munava \emph{u10624610} \newline
Paul Engelke \emph{u13093500} \newline
New Member \emph{uxxxxxxxx} \newline
New Member \emph{uxxxxxxxx} \newline
New Member \emph{uxxxxxxxx} \newline
New Member \emph{uxxxxxxxx} \newline 
\newline 
\newline
\end{flushright}
%\end{minipage}
GitHub: \href{https://github.com/davidbreetzke/COS_301_Phase2_1B}{Visit}
\vfill
{\large Version 1}
\\
{\large \today}

\end{center}
\end{titlepage}


\tableofcontents	%Creates Table of contents from sections and subsections, etc...
\newpage

%Access and Integration Requirements
	\newpage
	\begin{center}
	\section{\textbf{\huge{Access and Integration Requirements}}}
	\end{center}
	
	[insert text]
%---------------------------------------------------------

%Access Channels
	\newpage
	\begin{center}
	\section{\textbf{\huge{Access Channels}}}
	\end{center}
	
	[insert text]
%---------------------------------------------------------
		
%Integration Channels
	\newpage
	\begin{center}
	\section{\textbf{\huge{Integration Channels}}}
	\end{center}
	
	[insert text]
%---------------------------------------------------------

%Architectural Responsibilities
	\newpage
	\begin{center}
	\section{\textbf{\huge{Architectural Responsibilities}}}
	\end{center}
	
	\textbf{Responsibilities which need to be addressed by the software architecture are:}
	\\
	
	\begin{itemize}
	\item To be able to host and provide an environment for the execution of the system
	
	\item To provide an Infrastructure that provides a web access channel
	
	\item To provide an Infrastructure that provides a mobile access channel
	
	\item To integrate with the LDAP repository
	
	\item To provide an infrastructure that allows module plug-ability
	
	\item Provide infrastructure that allows in-dependency from core modules and add on modules
	
	\item To provide an infrastructure to integrate the system into the department's website
	
	\item To provide an infrastructure to integrate the Hamster marking system
	\end{itemize}
%---------------------------------------------------------

%Quality Requirements
	\newpage
	\begin{center}
	\section{\textbf{\huge{Quality Requirements}}}
	\end{center}
%---------------------------------------------------------
	\subsection{Scalability}
		[insert text]
	\subsection{Performance Requirements}
		[insert text]
	\subsection{Maintainability}
		[insert text]
	\subsection{Reliability and Availability}
		[insert text]
	\subsection{Security}
		[insert text]
	\subsection{Monitorability and Auditability}
		[insert text]
	
	\subsection{Scalability}	
		[insert text]
	
	\subsection{Testability}
	
		\subsubsection{Type of Quality:}
			\textbf{} System Quality
		
		\subsubsection{Priority:}
			\textbf{} Critical
		
		\subsubsection{Description:}
			\textbf{}Testability measures how easy it is to create testing standards for a system and its individual components, theses standards are tested to evaluate if a criteria has been met. Thus software testability is the point to which the software system supports testing in some context. Hence if the software testability is high finding faults in the system is easier.
		
		\subsubsection{Stake Holder:}
		\begin{itemize}
			\item Persons who operates the system : Administrator, Maintenance Operator and Tech-team.
			\item Persons who benefits from the system : Lectures, Teaching Assistance, Tutors, Students and Guest.
			\end{itemize}
		\subsubsection{Context:}
		\begin{itemize}
			\item Stimulus : The testing is performed by tester (these might be system testers, integration testers and even the end user).
			\item Artifact : The target of the attack can be the system or the data in the system.
			\item Environment – This attack can come from the user of the system or an outsider like a hacker. 
			\item Response : The system has to authorize certain actions and responses for each of the given tasks.
			\item Response Measure : The measure of the system and it functionality before, during and after the attack.
		\end{itemize}
		
		
		\subsubsection{Measurable Specification:}
		\begin{itemize}
			\item Understand-ability : The point at which the component of the system that being tested is self-explanatory.
			\item Separation of concerns : The point, at which the component of the system that's being tested has a well-defined responsibility.
			\item Observe-ability : The point, at which the component of the system that’s being tested become possible to discern the test results. \\
						
			\textbf{Component Under Test}\\
			Is a test that restrictions the scope of the used software to a ration of the system that is being tested.
			
			\item Controllability : The point, at which the system that’s being tested becomes possible to control the state of the component under test as required.
			\item Isolate-ability : The point, at which the system that’s being tested becomes possible for the component under test to be tested in isolation.
			
		\end{itemize}
		
		
	\subsection{Usability }
		
		\subsubsection{Type of Quality:}
			\textbf{} User Quality
		
		\subsubsection{Priority:}
		\textbf{} Critical
		
		\subsubsection{Description:}
			\textbf{}Usability describes how the system meets the requirements of the stake holders by being instinctive on condition that good access for incapacitated users is provided, and resulting overall great user experience. Thus software usability refers to the ease of use and learn-ability of the system. In other words how user-friendly is it.  
		
		\subsubsection{Stake Holder:}
			\begin{itemize}
				\item Persons who benefits from the system : Lectures, Teaching Assistance, Tutors, Students and Guest.
			\end{itemize}
		\subsubsection{Context:}
			\begin{itemize}
				\item Stimulus : The stake holder wants to use the system efficiently.
				\item Artifact : The target of use which is the system.
				\item Environment : This stake holder’s action with which the usability quality is concerned.
				\item Response : The system provides the stake holder with features that the stake holder will or might need.
				\item Response Measure : The response of the system and it functionality is measured by the number of errors, number of problems encountered, user satisfaction and time taken per task.
			\end{itemize}
				
		\subsubsection{Measurable Specification:}
			\textbf{Cognitive Modelling Methods}\\
				Cognitive Modelling Methods involves creating computational method in order to estimate the time it will take people to perform given tasks.
			\begin{itemize}			
			
				\item Human Processor Model : This model was developed to calculate how long it takes an individual to perform a task. A table is given with amount of times a user would take to execute an action i.e. move eye to look at the screen 230ms.
				\item Keystroke level modelling : Very much like the GOMS version but simplifies assumptions so that calculation time and complexity is reduced.
				\item  Heuristic Evaluation	 : This measurable method involves bringing in a set of experts that will evaluate the usability of your system based on their prior knowledge and research. 
			
			\end{itemize}
		
	\subsection{Integrability}
		
		\subsubsection{Type of Quality:}
			\textbf{} User Quality
		
		\subsubsection{Priority:}
		\textbf{} Critical
		
		\subsubsection{Description:}
			\textbf{}The capability of making components of a single system that is isolated and developed separately work together correctly.
		
		\subsubsection{Stake Holder:}
			\begin{itemize}
				\item Persons who operates the system : Administrator, Maintenance Operator and Tech-team.
			\end{itemize}
		\subsubsection{Context:}
			\begin{itemize}
				\item Stimulus : The stake holder wants to use to configure, maintain, update and use the system.
				\item Artifact : The target of use which is the system.
				\item Environment : This stake holder’s action with which the system has to conform to.
				\item Response : The system allows the stake holders to interact with it.
				\item Response Measure : The response of the system and it functionality after the system has 				  been refactored.
			\end{itemize}
				
		\subsubsection{Measurable Specification:}
			
			\begin{itemize}			
			
				\item Spread load across time : This can be achieved by making use of Queuing as a tactic.
				\item Reduce communication load : This can be accomplished by using strategies like compression, batching and course grained services.
				\item  Fault prevention : This can be addressed by making use of persistent messaging.
				\item Component Application  : We encounter naming service, trader service and interface/ contract repository which is addressed with integer-ability.
				\item  Security : The use of encryption and restricting accessibility will address the security insures we might come across.
			
			\end{itemize}
			
			\subsubsection{Required Integration Channels:}
			\begin{itemize}			
			
				\item The Required Integration Channels are those that the system will require to make it accessible by the stake holders.
				\begin{itemize}			
			
					\item Graphical User Interface.
				
				\end{itemize}	
			
				\item The Required Integration Channels are those that the server will host the system.
			\end{itemize}	
			\subsubsection{Protocol Requirements on Integration Channels:}
			\begin{itemize}			
			
				\item HTTP (REQUEST/GET/POST requests)
				\item SOAP (Messaging Protocol)
				
			\end{itemize}	
			
			\subsubsection{Quality requirements on integration channels:}
			\begin{itemize}			
			
				\item ISDN (Simultaneous digital transmission of data, video and voice)
				
			\end{itemize}	

%Architecture Constraints
	\newpage
	\begin{center}
	\section{\textbf{\huge{Architecture Constraints}}}
	\end{center}
		\subsubsection{Description:}
		\textbf{}Below are system constraints that have a significant bearing on the Buzz system architecture.
		These constraints are divided into two categories.
		
		
	
	\subsubsection{Technical constraints:}
	\begin{itemize}
		\item These are fixed technical design decisions that absolutely cannot be changed.
	\end{itemize}
	\subsubsection{Business constraints:}
	\begin{itemize}
		\item These are unchangeable business decisions that in some way restrict the software architecture.
	\end{itemize}
	
	\section{Technical Constraints:}
		\subsubsection{	Programming technologies:}
			\textbf{}Buzz system will primarily be implemented using various Java view technologies and frameworks. These technologies offer important functionality for distributed web applications and should provide strong foundation for Buzz.
		
		
	
			\textbf{}Java technologies to be used include:
   	\begin{itemize}		
			\item Java EE
			\item JPA(Java Persistence API)
			\item JPQL(Java  Persistence Query Language)
	 		\item JSF(Java Server Faces)
	\end{itemize}
		\textbf{} Other technologies include: HTML, AJAX and CSS
			\begin{itemize}		
			
		\item 	SOAP (Simple Object Access Protocol) will be used as interface for exchanging information on the web services and across the network.
		\item 	UTF-8 must be used for encoding to keep information safe and secure.
		
		\item 	GitHub will be used for a web based repository to store all of the information regarding the project.
			\end{itemize}
			\subsubsection{Platforms supported and operating system:}
		
			
			\textbf{}The CS (Computer science) systems operate mainly on Linux. Buzz should also support Linux as it will need to be integrated with other existing systems such as:
			\begin{itemize}		
				\item	Computer Science LDAP repository
				\item 	Computer Science Portal
				\item 	World Wide Web
			\end{itemize}
			\textbf{}Buzz should support major browser clients that include:
			\begin{itemize}		
					\item		Mozilla Firefox
					\item		Google Chrome
					\item		Safari
					\item		Internet Explorer
				
			\end{itemize}
			\textbf{}Buzz will not support the Android platform because of the limited time allocated for the development.
			\subsubsection{Hardware:}
			
			
			\textbf{}Buzz will be hosted by the computer science server and therefore should support the server’s hardware capabilities while providing the recommended functionality to the users.
			
			\subsubsection{Use of a specific library or framework:}
			
			
			\textbf{}Any specific framework or library can be used/should be used for Buzz provided that it’s open source.
	
			\section{ Business Constraints:}
				\subsubsection{Schedule:}
				
				
				\textbf{}Buzz System must be implemented within the stipulated time period and should be functional at the end of the deadline.
					\subsubsection{Budget:}
					
					
					\textbf{}A budget has not been allocated for buzz system and therefore its recommended that any technology or product that requires capital should not be used.
						\subsubsection{Software licensing restrictions:}
						
						
						\textbf{}Buzz should adhere to all software licensing restrictions that are stipulated by the software owners.
%---------------------------------------------------------
\end{document}